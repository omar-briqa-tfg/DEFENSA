\section{Conclusions}\label{sec:conclusions}

\begin{frame}{Conclusions I}

\begin{itemize}
    \item El projecte ha assolit plenament l’objectiu inicial: transformar dels registres d’accés amb un format específic del programari d’UPCommons a un conjunt de dades comprensibles.
    \item Hem posat el focus en la modularitat del disseny de l'eina, i no acoblar aquest amb la implementació.
    \item Hem pogut descartar i no emmagatzemar un 22\% del logs (430.789.238 registres).
\end{itemize}

\end{frame}

\begin{frame}{Conclusions II}

    \begin{itemize}
        \item Utilitzem el protocol OAI-PMH, i la llibreria Sickle.
        La complexitat d’aquest protocol fa possible utilitzar peticions d’HTTP directes contra el servidor OAI, ometent l’ús de llibreries externes.
        \item Tot aquest processament s’ha dut a terme en el servidor específic pel desenvolupament del projecte.
        \item El fet de disposar la base de dades en una altra ubicació, ja sigui en un altre servidor intern fora de la xarxa local, o en un servei al núvol, té certes implicacions
    \end{itemize}

\end{frame}

\begin{frame}{Conclusions III}

    \begin{itemize}
        \item Com ja hem vist als diferents casos d’ús, els paràmetres escollits per processar els logs i emmagatzemar-los ens han permès dur algun tipus d’anàlisis de forma immediata.
        \item Grafana ens ha permès analitzar visualment aquests casos, obtenint informació de les nostres fonts de dades.
    \end{itemize}

\end{frame}

\begin{frame}{Línies futures de treball}

    \begin{itemize}
        \item A
        \item B
    \end{itemize}

\end{frame}

\begin{frame}{Final}
    Arribats a aquest punt, hem vist com a partir d’analitzar els registres d’accés a UPCommons, entendre el funcionament de les metadades dels recursos, afegir aquesta informació a un gestor de bases de dades, hem pogut desenvolupar uns casos d’ús base que demostren que aquesta eina de codi obert pot donar suport per analitza les diferents casuístiques proposades per l’usuari de l’eina.
\end{frame}