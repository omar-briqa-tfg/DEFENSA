\section{Emmagatzematge de les dades}\label{sec:data-storing}

\begin{frame}{Format dels \textit{logs}}

    \begin{itemize}
        \item Aquests són els camps principals d'interés que obtenim directament dels \textit{logs}.
        \item El pas dels anys (i el canvi de les versions de DSpace) no han fet que canviï gaire el format.
        \begin{itemize}
            \item \texttt{ip\_address}
            \item \texttt{time}
            \item \texttt{request}
            \begin{itemize}
                \item \texttt{method}
                \item \texttt{resource}
                \item \texttt{version}
                \item \texttt{status\_code}
                \item \texttt{response\_size}
            \end{itemize}
            \item \texttt{referer}
            \item \texttt{user\_agent}
        \end{itemize}
    \end {itemize}
\end{frame}


\begin{frame}{Base de dades pels logs}
    \begin{itemize}%[<+- | alert@+>]
        \item Característiques:
        \begin{itemize}
            \item Base de dades de codi obert.
            \item Enfocada a registres d'accés.
            \item Ben documentada, actualitzada i mantinguda per una comunitat activa d'usuaris.
            \item Capacitat per processar grans quantitats de dades.
            \item Capacitat d'allotjar la nostra base de dades localment al nostre servidor de treball.
        \end{itemize}
        \item Hem considerat diverser opcions: Loki, Elastic, SOLR, InfluxDB, etc.
        \item Decisió final: InfluxDB.
    \end{itemize}
\end{frame}

\begin{frame}{InfluxDB}

    \begin{itemize}%[<+- | alert@+>]
        \item Base de dades de codi obert compatible per diferents plataformes.
        \item És una eina ben documentada, mantingut activament i de forma actual.
        \item Integració per defecte amb Grafana per poder visualitzar els \textit{logs}.
        \item Diverses llibreries ja implementades per enviar \textit{logs} a la base de dades.
        \begin{itemize}
            \item En el nostre cas emprar \textit{influxdb-client} pel llenguatge \textit{Python}.
        \end{itemize}
    \end{itemize}
\end{frame}


\begin{frame}{Paràmetres addicionals d'InfluxDB}
    \begin{itemize}%[<+- | alert@+>]
        \item \texttt{tag set}: conjunt de parelles clau-valor \textbf{indexades}.
        \begin{itemize}
            \item \textbf{content}: format del contingut, pot ser:
            \begin{itemize}
                \item \texttt{ok}: el format del \textit{log} és l'esperat.
                \item \texttt{diferent}: el format del \textit{log} no és l'esperat, però s'ha pogut tractar.
                \item \texttt{error}: error de processament.
            \end{itemize}
            \item \textbf{method}: mètode HTTP present al \textit{log}.
            \item \textbf{status\_code}: codi d'estat de la resposta de la petició d'HTTP.
            \item \textbf{type}: tipus de \textit{log}, pot ser:
            \begin{itemize}
                \item \texttt{cerca}: cerca a la plataforma d'UPCommons.
                \item \texttt{recurs}: accés a un recurs d'UPCommons.
                \item \texttt{altres}: altres tipus de \textit{log}.
            \end{itemize}
        \end{itemize}
        \item \texttt{field set}: conjunt de parelles clau-valor \textbf{no indexades}.
        \begin{itemize}
            \item \textbf{log}: registre complet d'UPCommons.
            \item \textbf{recurs}: identificador handle, en cas que es tracti d'un accés a un recurs.
        \end{itemize}
    \end{itemize}
\end{frame}


\begin{frame}{Resultat de l'abocament dels logs I}
    \begin{table}[!t]
        \label{tab:table1}
        \centering
        \begin{tabular}{@{} lr @{}}
            \toprule
            \textbf{2006-2023} & Total\\
            \midrule
            Nº total de \textit{logs}       & 1.922.392.760\\
            Accés a recursos UPCommons      & 1.208.133.268\\
            Cerques a UPCommons             & 179.937.769\\
            Accés a recursos web descartats & 430.789.238\\
            Altres tipus de \textit{logs}   & 107.198.724\\
            Errors de processament          & 552\\
            Duració (minuts)                & 1.977,13\\
            \bottomrule
        \end{tabular}
    \end{table}
\end{frame}

\begin{frame}{Resultat de l'abocament dels logs II}

    \begin{figure}
        \begin{tikzpicture}
            \begin{axis}[
                mlineplot,
                width=\textwidth,
                height=0.7\textheight,
                ylabel=Logs,
                xtick={2006, 2008, 2010, 2012, 2014, 2016, 2018, 2020, 2022},
                xticklabel style={/pgf/number format/1000 sep=},
                yticklabel style={/pgf/number format/fixed,/pgf/number format/precision=1},
                legend style={at={(0.5,-0.2)}, anchor=north, legend columns=2, /tikz/every even column/.append style={column sep=0.5cm}}
            ]
                \addplot coordinates {
                    (2006, 1553768)
                    (2007, 14970518)
                    (2008, 30747039)
                    (2009, 37399831)
                    (2010, 76099876)
                    (2011, 109822586)
                    (2012, 109646515)
                    (2013, 70230167)
                    (2014, 68059608)
                    (2015, 146599579)
                    (2016, 139559925)
                    (2017, 133645398)
                    (2018, 140353396)
                    (2019, 127218126)
                    (2020, 166038130)
                    (2021, 192986484)
                    (2022, 189863676)
                    (2023, 167515925)
                };
                \addlegendentry{Nº total de \textit{logs}}
                \addplot coordinates {
                    (2006, 239410)
                    (2007, 3039294)
                    (2008, 8563620)
                    (2009, 17399497)
                    (2010, 44754685)
                    (2011, 78153547)
                    (2012, 80527672)
                    (2013, 33835954)
                    (2014, 28482567)
                    (2015, 97702657)
                    (2016, 95919593)
                    (2017, 80272838)
                    (2018, 90823546)
                    (2019, 83529395)
                    (2020, 99891136)
                    (2021, 124302571)
                    (2022, 126879940)
                    (2023, 113815346)
                };
                \addlegendentry{Recursos d'UPCommons}
                \addplot coordinates {
                    (2006, 584614)
                    (2007, 2043778)
                    (2008, 7421858)
                    (2009, 3987884)
                    (2010, 12385336)
                    (2011, 12493351)
                    (2012, 11728372)
                    (2013, 16093356)
                    (2014, 14118753)
                    (2015, 16117702)
                    (2016, 11252974)
                    (2017, 14075887)
                    (2018, 12055214)
                    (2019, 6815726)
                    (2020, 11351531)
                    (2021, 10237352)
                    (2022, 10820565)
                    (2023, 6353516)
                };
                \addlegendentry{Cerques a UPCommons}
                \addplot coordinates {
                    (2006, 325932)
                    (2007, 7997613)
                    (2008, 10402957)
                    (2009, 11133194)
                    (2010, 9946743)
                    (2011, 11428684)
                    (2012, 13582288)
                    (2013, 17432325)
                    (2014, 21320545)
                    (2015, 24590560)
                    (2016, 24606115)
                    (2017, 28598003)
                    (2018, 32576258)
                    (2019, 32556241)
                    (2020, 48773842)
                    (2021, 51119990)
                    (2022, 48192917)
                    (2023, 36205031)
                };
                \addlegendentry{Recursos web descartats}
            \end{axis}
        \end{tikzpicture}\label{fig:log-push-results}
    \end{figure}

\end{frame}


\begin{frame}{Resultat de l'abocament dels logs III}

    \begin{figure}
        \begin{tikzpicture}
            \begin{axis}[
                mlineplot,
                width=\textwidth,
                height=0.7\textheight,
                ylabel=Minuts,
                xtick={2006, 2008, 2010, 2012, 2014, 2016, 2018, 2020, 2022},
                xticklabel style={/pgf/number format/1000 sep=},
                yticklabel style={/pgf/number format/fixed,/pgf/number format/precision=1},
                legend style={at={(0.5,-0.2)}, anchor=north, legend columns=2, /tikz/every even column/.append style={column sep=0.5cm}}
            ]
                \addplot coordinates {
                    (2006, 1.6158425967809)
                    (2007, 11.2992274621667)
                    (2008, 28.9930860479672)
                    (2009, 38.2631061474481)
                    (2010, 88.525434136391)
                    (2011, 123.040361817678)
                    (2012, 121.413944741090)
                    (2013, 70.0155762513478)
                    (2014, 63.0623456716536)
                    (2015, 160.5436228156080)
                    (2016, 150.4153102318440)
                    (2017, 134.3766959190370)
                    (2018, 139.3660171588260)
                    (2019, 126.8452290733650)
                    (2020, 159.0830255985260)
                    (2021, 189.5334932247790)
                    (2022, 194.7787342015900)
                    (2023, 175.9636698087050)
                };
                \addlegendentry{Durada de l'abocament}
            \end{axis}
        \end{tikzpicture}\label{fig:log-push-time}

    \end{figure}

\end{frame}

\begin{frame}{Base de dades per les metadades}
    \begin{itemize}%[<+- | alert@+>]
        \item Característiques:
        \begin{itemize}
            \item Quatre ordres de magnitud inferior que la dels \textit{logs}.
            \item Els requisits no són tan estrictes
            \item Flexibilitat del model de dades
        \end{itemize}
        \item Decisió final: MongoDB.
        \begin{itemize}
            \item Base de dades no relacional de codi obert, orientada a documents.
            \item Com alguns camps no són únics, necessitem un gestor que ens ofereixi flexibilitat al seu esquema de dades.
            \item Com el servidor OAI retorna les dades en format XML, la conversió a un document JSON és quasi trivial.
        \end{itemize}
        \item Hem fet primer l'abocament al servidor de treball local, després a la base de dades.
    \end{itemize}
\end{frame}


\begin{frame}{Resultat de l'abocament de les metadades}
    \begin{table}[!t]
        \label{tab:table2}
        \centering
        \begin{tabular}{|c|c|c|}
            \hline
            MongoDB & Nº total de metadades & Duració (segons)\\
            \hline
            \textbf{Total} & 242.555 & 38,2050559520721\\
            \hline
        \end{tabular}
    \end{table}

    \begin{table}[!t]
        \label{tab:table3}
        \centering
        \begin{tabular}{|c|c|c|}
            \hline
            Servidor de treball & Nº total de metadades & Duració (minuts)\\
            \hline
            \textbf{Total} & 242.555 & 20,95360236565\\
            \hline
        \end{tabular}
    \end{table}
\end{frame}